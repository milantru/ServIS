\chapter{Vývojová dokumentácia}

V~tejto kapitole sa pozrieme na~implementáciu nášho~systému. Zameráme sa na~organizáciu a~implementáciu dôležitých častí aplikácie. Cieľom tohto textu nie je opísať správanie každého riadku kódu. Na~to slúži kód samotný~(prípadne komentáre, ktoré autor pridal na~miesta, kde to uznal za~vhodné). Odkaz k~zdrojovým kódom aplikácie sa nachádza v~prílohe~(\ref{implementacia}).

Celá aplikácia je rozdelená do~dvoch projektov-~ServISData a~ServISWebApp.

\section{ServISData}

Zmyslom projektu ServISData je správa databázových entít a~komunikácia s~databázou. Teraz si prejdeme jednotlivé priečinky tohto projektu a~opíšeme si ich~obsah.

\subsection{Koreňový priečinok projektu}

Nachádza sa tu \verb|ServISDbContext| a~spolu s~ním i~\verb|ServISDbContextFactory|. Tieto triedy sú zodpovedné za~konfiguráciu databázy, a~tiež samotné pripojenie aplikácie k~databáze. Viac o~týchto triedach a~spôsobu ich použitia v~projekte nájdeme~\href{https://learn.microsoft.com/en-us/ef/core/dbcontext-configuration/\#using-a-dbcontext-factory-eg-for-blazor}{tu}.

V~priečinku sa tiež nachádzajú triedy~\verb|AutogeneratedMessageForExtensions| a~\verb|InputTypeExtensions|. V~oboch prípadoch ide o~extension metódy umožňujúce získať metadáta z~atribútov, ktoré sme použili\\v~enumoch~\verb|AutogeneratedMessage.For| a~\verb|InputType|.

Takisto sa v~priečinku nachádza už spomenutý~\verb|InputType|. Ide o~enum, ktorého hodnoty predstavujú možné typy hodnôt vlastností stroja.

Ďalej sa tu nachádza trieda~\verb|ServISApi|. Tá slúži pre~komunikáciu s~databázou. Respektíve umožnuje iným projektom~(v~našom prípade ServISWebapp) modely ukladať, čítať, editovať a~mazať z~databázy.

\subsection{Attributes}

Priečinok obsahuje atribúty\footnote{\url{https://learn.microsoft.com/en-us/dotnet/csharp/advanced-topics/reflection-and-attributes/}}. Konkrétne ide\\o~\verb|AutogeneratedMessageDataAttribute| a~\verb|InputTypeLabelAttribute|.

Prvý slúži na~nastavenie predvoleného predmetu a~textu automaticky generovaných správ, ale takisto aj na~nastavenie ich podporovaných tagov.

Druhý slúži pre~uloženie užívateľsky prívetivejšieho názvu typu hodnoty vlastnosti stroja. Tieto názvy sa zobrazujú napríklad pri~vytváraní typu vlastnosti stroja.

\subsection{DataOperations}
\label{dataoperations}

Priečinok obsahuje triedy, ktoré ich~užívateľom umožnia vykonávať rôzne operácie nad~dátami (filtrovanie, stránkovanie,\dots).

\subsection{Interfaces}

Priečinok obsahuje rozhrania~(ang.~interfaces).

\verb|IServISApi| nás zbaví potreby upravovať kód využívajúci API projektu ServISData v~prípade, ak sa rozhodneme vymeniť \verb|ServISApi| za~nejakú inú implementáciu.

\verb|IPhoto| nám umožní všeobecne pracovať s~fotkami aj napriek tomu, že pôjde o~fotky rôznych entít.

\verb|IItem| nám dovolí písať všeobecný kód pre~prácu s~modelmi entít~(využíva sa napr.~pri~mazaní entít z~databázy; viď~metódu~\verb|DeleteItem| v~triede~\verb|ServISApi|).

\subsection{Migrations}

Tento priečinok bol vygenerovaný frameworkom~Entity~Framework~Core a~obsahuje vygenerovaný kód. Ide o~migrácie\footnote{\url{https://learn.microsoft.com/en-us/ef/core/managing-schemas/migrations/?tabs=dotnet-core-cli}}. Nad~migráciou môžeme rozmýšlať ako nad~commitom v~gitu. Po~aplikovaní migrácie dôjde k~zmene v~databáze~(napr.~sa pridá nový stĺpec do~nejakej z~tabuliek).

\subsection{Models}
\label{models}

Priečinok obsahuje modely-~triedy reprezentujúce entity uložené v~databáze.

Špeciálne sa zastavíme pri~triede~\verb|AutogeneratedMessage|\\a~enume~\verb|AutogeneratedMessage.For|. Platí, že každá hodnota enumu predstavuje druh automaticky generovanej správy.\\Napríklad \verb|AutogeneratedMessage.For.AutionWinner| predstavuje automaticky generovanú správu pre~víťaza aukcie. A~tiež platí, že pre~každú hodnotu tohto enumu existuje inštancia triedy~\verb|AutogeneratedMessage| uložená v~databáze.\\Pred\-vo\-le\-né hodnoty správ sa nachádzajú v~atribútoch hodnôt enumu a~administrátormi definované hodnoty sa ukladajú do~databázy.

\section{ServISWebApp}

Projekt ServISWebApp predstavuje rozhranie pre~užívateľov a~zároveň obsahuje logiku nášho~systému. Znova, ako v~predošlej podkapitole, si prejdeme jednotlivé priečinky tohto projektu a~opíšeme si ich~obsah.

\subsection{Koreňový priečinok projektu}

Priečinok obsahuje súbory~\_Imports.razor~(nachádzajú sa v~ňom \verb|using| direktívy aplikované pre~každý komponent v~projekte), App.razor~(koreňový komponent), appsettings.json~(ide o~konfiguráciu aplikácie, v~ktorej okrem iného vieme nastaviť emailovú adresu, ktorú systém využíva pre~odosielanie a~prijímanie emailov) a~Program.cs~(obsahuje kód zodpovedný za~spustenie celej aplikácie). Pre~viac informácií o~štruktúre Blazor projektu môže čítateľ kliknúť~\href{https://learn.microsoft.com/en-us/aspnet/core/blazor/project-structure?view=aspnetcore-7.0}{tu}.

V~časti Models~\ref{models} bolo spomenuté, že pre~každú hodnotu\\enumu~\verb|AutogeneratedMessage.For| existuje inštancia triedy\\\verb|AutogeneratedMessage| uložená v~databáze. To zabezpečuje metóda\\\verb|StoreNewAutogeneratedMessagesAsync|, ktorá sa nachádza v~Program.cs.

\subsection{Auth}

Priečinok obsahuje triedy~zodpovedné za~prihlasovanie užívateľa a~hashovanie hesla.

\subsection{BackgroundServices}

V~tomto priečinku sa nachádzajú triedy~predstavujúce služby bežiace na~pozadí. Konkrétne ide o~službu zodpovednú za~vyhodnocovanie aukcie a~službu, ktorá každú sekundu vykoná akciu, ktorú si užívateľ triedy v~službe zaregistroval~(v~našom prípade ide o~prerenderovanie komponentov pri~odpočte).

\subsection{Components}

V~tomto priečinku sa nachádzajú komponenty využité v~aplikácii.

Komponenty ktorých meno končí na~slovo~``Lister'', slúžia na~zobrazenie~(vylistovanie) nejakého obsahu. Väčšinou sú tým obsahom komponenty končiace na~slovo~``Card''.

Ďalej sa tu nachádzajú komponenty začínajúce na~slovo~``Input''. Tieto komponenty spolu s~komponentom~\verb|ChecklistTable| slúžia na získavanie vstupu užívateľa.

Komponenty~\verb|Message|, \verb|Messages|, \verb|ThreadRow|, \verb|ThreadView| a~\verb|ThreadsView| tvoria miesto pre manažovanie správ. Umožňujú adminovi čítať správy, a~takisto na~nich odpovedať. Navyše~\verb|MessageSettings| umožňuje adminovi meniť tvar automaticky generovaných správ.

Komponent~\verb|CountdownDisplayer| sa využíva na~zobrazenie odpočtu do~konca aukcie.

\verb|CustomDataAdaptor| sa využíva v~komponente~\verb|ItemsManagement| na~čítanie dát.

\verb|LoginPanel| umožňuje neprihlásenému užívateľovi prístup k~prihlasovaciemu a~registračnému formuláru. Alebo v~prípade prihláseného užívateľa poskytuje možnosti prejsť na~profil a~odhlásiť sa.

\verb|PhotoSlider| sa využíva na~zobrazovanie obrázkov~(môžeme ho vidieť napríklad na~stránke detailu ľubovoľného stroja).

Komponent~\verb|TabControl| v~sebe obsahuje komponenty~\verb|TabPage|. Umožňujú užívateľom~(najmä adminom, pretože bežní zákaznici majú iba jeden tab) preklikávať sa medzi rôznymi časťami profilu.

Ďalej sa tu nachádzajú priečinky~Buttons, Forms a~Managements.

V~priečinku~Buttons sa nachádzajú komponenty reprezentujúce rôzne tlačidlá.

Vo~Forms sa nachádzajú rôzne formuláre ktorými dokážeme vytvárať a~do~databázy ukladať inštancie modelov~(napr.~stroje, prídavné zariadenia,\dots). Ale takisto sa tam nachádza i~komponent \verb|DemandForm|, ktorý zákazníkom umožňuje odosielať dopyt.

A~v~priečinku Managements sa nachádza komponent~\verb|ItemsManagement|. Tento komponent tvorí rozhranie pre~administrátorov na~správu entít~(napr.~strojov, náhradných dielov,\dots). Umožňuje adminom entity vytvárať, editovať a~mazať. Takisto im pri~prezeraní existujúcich entít umožňuje ich vyhľadávanie a~triedenie.

\subsection{CssProviders}

Triedy v~tomto priečinku sú zodpovedné za~zmenu css štýlov vo~formulároch pri~modifikovaní a~validácií políčok formuláru. O~triede~\verb|FieldCssClassProvider| sa čítateľ môže dozvedieť viac \href{https://learn.microsoft.com/en-us/dotnet/api/microsoft.aspnetcore.components.forms.fieldcssclassprovider?view=aspnetcore-7.0}{tu}.

\subsection{Pages}

V~tomto priečinku sa nachádzajú stránky a~priečinok~Admin, ktorý obsahuje stránky exkluzívne pre~administrátorov.

\subsection{Resources}

Tento priečinok obsahuje ``.resx'' súbory so~slovenskými prekladmi slov a~viet. Využívajú ich komponenty z~balíčku od~spoločnosti Syncfusion.

\subsection{Shared}

V~tomto priečinku sa nachádzajú triedy zdieľané (potenciálne) celým projektom.

Triedy~\verb|Email| a~\verb|Thread| slúžia ako dátonosné alternatívy k~triedam balíčka MailKit. Môžeme nad~nimi rozmýšľať ako nad fasádami, ktoré zjednodušujú prístup k~dátam.

\verb|EmailManager| pokrýva funkcionalitu práce s~emailami, napr.~ich posielanie a~prijímanie.

\verb|MainLayout| a~\verb|NavMenu| sú štandardné Blazor komponenty. Prvý definuje rozloženie stránky a~druhý navigáciu.

\verb|SyncfusionLocalizer| a~\verb|SyncfusionDataOperations| sú triedy, ktoré využívajú kód z~balíčka od~spoločnosti~Syncfusion. Prvá sa stará o~nastavenie lokalizácie užívateľského rozhrania~Syncfusion komponentov. Druhá vykonáva operácie nad~dátami~(napr.~vyhľadávanie, triedenie,\dots). Čítateľ môže byť v~tento moment mierne zmätený, pretože raz sme tu už podobnú triedu mali. Konkrétne v~časti DataOperations~(\ref{dataoperations}). Rozdiel je v~tom, že trieda spomenutá v~predošlej časti využíva na~konfiguráciu operácií našu vlastnú triedu\\\verb|MyDataOperations.Configuration|, kdežto \verb|SyncfusionDataOperations|\\využíva triedu~\verb|DataManagerRequest|, ktorá je takisto z~balíčka spoločnosti~Syncfusion.

\verb|FileTools| združuje metódy pre~prácu so~súbormi. Konkrétne v~našom prípade ho využívame pre~prácu s~fotkami.

Ostali nám už len trieda~\verb|AuctionSummary| a~trieda\\\verb|AutogeneratedMessageExtensions|, ktorá sa nachádza v priečinku Extensions. Obe triedy spolu úzko súvisia.\\\verb|AutogeneratedMessageExtensions| obsahuje extension metódy pre~triedu\\\verb|AutogeneratedMessage|. \verb|AutogeneratedMessage| obsahuje text (predmet a telo správy). V~texte sa nachádzajú tagy a~slúži ako šablóna. Metódy triedy\\\verb|AutogeneratedMessageExtensions| fungujú tak, že vezmú túto šablónu a~namiesto tagov dosadia skutočné dáta vzaté z~inštancií\\triedy~(príp.~tried)~\verb|AuctionSummary|.
