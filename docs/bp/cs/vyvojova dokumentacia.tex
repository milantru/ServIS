\chapter{Vývojová dokumentácia}

V~tejto kapitole sa pozrieme na~implementáciu nášho~systému. Zameráme sa na~organizáciu a~implementáciu dôležitých častí aplikácie. Cieľom tohto textu nie je opísať správanie každého riadku kódu. Na~to slúži kód samotný~(prípadne komentáre, ktoré autor pridal na~miesta, kde to uznal za~vhodné).

Celá aplikácia je rozdelená do~dvoch projektov-~ServISData a~ServISWebApp.

\section{ServISData}

Zmyslom projektu ServISData je správa databázových entít a~komunikácia s~databázou. Teraz si prejdeme jednotlivé priečinky tohto projektu a~popíšeme si ich~obsah.

\subsection{Koreňový priečinok projektu}

Nachádza sa tu \verb|ServISDbContext| a~spolu s~ním i~\verb|ServISDbContextFactory|. Tieto triedy sú zodpovedné za~konfiguráciu databázy, a~tiež samotné pripojenie aplikácie k~databáze. Viac o~týchto triedach a~spôsobu ich použitia v~projekte nájdeme~\href{https://learn.microsoft.com/en-us/ef/core/dbcontext-configuration/\#using-a-dbcontext-factory-eg-for-blazor}{tu}.

V~priečinku sa tiež nachádzajú triedy~\verb|AutogeneratedMessageForExtensions| a~\verb|InputTypeExtensions|. V~oboch prípadoch ide o~extension metódy umožňujúce získať metadáta z~atribútov, ktoré sme použili v~enumoch~\verb|AutogeneratedMessage.For| a~\verb|InputType|.

Takisto sa v~priečinku nachádza už spomenutý~\verb|InputType|. Ide o~enum, ktorého hodnoty predstavujú možné typy hodnôt vlastností stroja.

Ďalej sa tu nachádza trieda~\verb|ServISApi|. Tá slúži pre~komunikáciu s~databázou. Respektíve umožnuje iným projektom~(v~našom prípade ServISWebapp) modely ukladať, čítať, editovať a~mazať z~databázy.

\subsection{Attributes}

Priečinok obsahuje atribúty\footnote{\url{https://learn.microsoft.com/en-us/dotnet/csharp/advanced-topics/reflection-and-attributes/}}. Konkrétne ide o~\verb|AutogeneratedMessageDataAttribute| a~\verb|InputTypeLabelAttribute|.

Prvý slúži na~nastavenie predvoleného predmetu a~textu automaticky generovaných správ, ale takisto aj na~nastavenie ich podporovaných tagov.

Druhý slúži pre~uloženie užívateľsky prívetivejšieho názvu typu hodnoty vlastnosti stroja. Tieto názvy sa zobrazujú napríklad pri~vytváraní typu vlastnosti stroja.

\subsection{DataOperations}

Priečinok obsahuje triedy, ktoré ich~užívateľom umožnia vykonávať rôzne operácie nad~dátami (filtrovanie, stránkovanie,\dots).

\subsection{Interfaces}

Priečinok obsahuje rozhrania~(ang.~interfaces).

\verb|IServISApi| nás zbaví potreby upravovať kód využívajúci API projektu ServISData v~prípade, ak sa rozhodneme vymeniť \verb|ServISApi| za~nejakú inú implementáciu.

\verb|IPhoto| nám umožní všeobecne pracovať s~fotkami aj napriek tomu, že pôjde o~fotky rôznych entít.

\verb|IItem| nám dovolí písať všeobecný kód pre~prácu s~modelmi entít~(využíva sa napr.~pri~mazaní entít z~databázy; viď~metódu~\verb|DeleteItem| v~triede~\verb|ServISApi|).

\subsection{Migrations}

Tento priečinok bol vygenerovaný frameworkom~Entity~Framework~Core a~obsahuje vygenerovaný kód. Ide o~migrácie\footnote{\url{https://learn.microsoft.com/en-us/ef/core/managing-schemas/migrations/?tabs=dotnet-core-cli}}. Nad~migráciou môžeme rozmýšlať ako nad~commitom v~gitu. Po~aplikovaní migrácie dôjde k~zmene v~databáze~(napr.~sa pridá nový stĺpec do~nejakej z~tabuliek).

\subsection{Models}

Priečinok obsahuje modely-~triedy reprezentujúce entity uložené v~databáze.

\section{ServISWebApp}
TBA
