\chapter{Návrh systému}

Kvôli P10~(\ref{dostupnost}) dáva dobrý zmysel vytvoriť náš systém ako webovú aplikáciu~-- vyriešime tým problém s distribúciou softvéru k (aj potenciálne novým) zákazníkom, a takisto vyriešime problém s potenciálne zastaralými počítačmi vo firme. Statická stránka by nám nestačila nakoľko chceme administrátorom umožniť dynamickú zmenu obsahu.

\section{Splnenie P2 a P3}

TBA: Pridal by som PP slide s hlavnou strankou, s kartickami, a detailom predmetov (vratane dopytoveho formulara) a opisal by som "Po prichode na stranku sa nam vylistuju karty hlavnej ponuky po prejdeni kurzorom na kartu hlavnej ponuky sa obrazok prepne na text (opis typu stroja)... Na obrazku X vidime aj formular na posielanie dopytu, ten sa otvori ked klikneme na tlacidlo Dopyt v Detaile, ktory vidime na obrazku Y...

\section{Splnenie P4}

TBA: Podobne ako v predchadzajucej sekcii... Znovu by sa v podstate opísali slidy. Možno by sa tiež hodilo spomenúť to, že keď sa skončí aukcia (už či s víťazom alebo bez), tak môže admin prísť a reštartovať, takze bolo by dobre vytvoriť separatny slide (zatiaľ neni) a ukázať na ňom rozdiel (rozdiel bude jedno tlačidlo, ak ešte beží tak bude písať uložiť a keď už skončila tka uložiť a reštartovať alebo tak nejak...)

\section{Splnenie P5}

TBA: podobne ako v predchadzajucej sekcii... Opisal slidy a spomenul predosle podmienky, ze prepojeni msa dostaneme na detaily (slidy spomenute uz boli v predoslych sekciach).

\section{Splnenie P6}

TBA: Podobne ako v predchadzajucej sekcii... Prilhasenie a Registraciu slidy spol us mojimi udajmi v profile co je ukazat a opisat.

\section{Splnenie P7}

TBA: podobne ako v predchadzajucej sekcii... TODO dorobit slide kde by bololepsie vidno tuto funckionalitu lebo zatial nemam

\section{Splnenie P8}

TBA: podobne ako v predchadzajucej sekcii...

\section{Splnenie P9}

TBA: podobne ako v predchadzajucej sekcii...

\section{Splnenie P1 a správy predmetov}

TBA: podobne ako v predchadzajucej sekcii by som opisal slidy ktore zobrazuju spravu (create, update, delete) predmetov... ALE s tym ze by som mozno spomenul ze vyuzijeme Syncfusion a ze to neni v rozpore s P11 (naklady) lebo existuje community verzia ktora je bezplatna. A taksito asi by som (momentalne nemam) dorobil verzie slidov, ktore uz boli predstavene v niektorych predoslych sekciach (karticky), v adminskom pohlade teda ze tam su tie tlacidla na vymazanie atd.

\section{Výber jazyka a frameworku}

TBA: Ako vidíme tak potrebujeme bohaty frontend, na to sa hodi napr JavaScriptové frameworky ako su React, Vue atd., a takisto nam nestaci Backend as a Service (ktory poskytuje napr Amazon...), lebo napr kvoli vyhodnocovaniu aukcie, takze preto vlastny backend. Na tvorbu webovej aplikacie s bohatym UI sa hodi high level jazyk, napr Java alebo C\#. Autor ovlada C\#, preto volime C\# a platformu .NET, ktora je s nim spojena. Backend by sme mohli implementovat pomocou frmeworku ASP.NET MVC. No este existuje ina varianta a tou je Blazor. Ten nam umozni vyuzivat komponenty (to sa nam hodi napr. aj kvoli implementácii odpočtu) a takisto nám umožní písať frontend aj backend v rovnakom jazyku -- C\#.

Blazor poskytuje viaceo hosting modelov a v dobe vyberu technologii existovali dva -- Blazor WebAssebmly a Blazor Server. Blazor WebAssembly funguje skor na sposob SPA, kde logika aplikacie bezi na klientskom pocitaci vo webovom prehliadaci, a su s nim spojene nejake problemy [vymenujem, napr to ze search engingy mozu mat problem s nim, P10 aby to uzivatelom rozbehol pc, aby nebol zastaraly prehliadac, prvotne nacitanie trva dlho (lebo stahuje zdrojaky) co by mohlo potencialnych zakaznikov odradit, a tiez nejake WebApi na server by sme museli vytvarat). Naproti Blazor Server sa podoba skor tradicnemu web app pristupu a hodi sa nam viac lebo je CEO-friendly, kod bezi na servery a uzivatel dostane len HTML, CSS, JS, preto aj starsie pc by nemali mat problem s rozbehnutim, a takisto nema preto problem s dlhym prvotnym nacitanim, a navyse nemusime vytvarata WebApi.

Preto si volime C\# (.NET) a Blazor Server.
