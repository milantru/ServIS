\chapter{Užívateľská dokumentácia}

Návod pre zákazníkov ako pracovať s~aplikáciou je možné nájsť v~podobe\linebreak videí na~adrese v~prílohe (viď~\ref{videonavod pre zakaznikov}), takisto je v~prílohe možné nájsť aj videonávody na~používanie aplikácie pre~administrátorov (viď~\ref{videonavod pre administratorov}). Čo sa týka procesov spustenia systému pre~lokálne testovanie a~nasadenia na~internet, tak tie si opíšeme v~tejto kapitole.

\section{Databáza}
\label{databaza}

Aplikácia ServIS využíva pre ukladanie dát databázu~MySQL, ktorá nie je súčasťou aplikácie, a~preto je nutné si ju osobitne stiahnuť a~nainštalovať. Odkaz na~stiahnutie je možné nájsť na~stránkach databázy~\cite{mysql}, rovnako ako aj návod na~inštaláciu~\cite{mysql navod}.

\section{Predpoklady pre lokálne spustenie}
\label{predpoklady}

Pre úspešné nainštalovanie, spustenie a~fungovanie aplikácie budeme potrebovať:

\begin{itemize}
\item Počítač s~operačným systémom Windows~10 alebo~vyššie
\item Prístup na internet
\item Visual Studio 2022~\cite{visual studio} (budeme potrebovať .NET~6 alebo~vyššie)
\item Databáza MySQL (viac v~podkapitole~\ref{databaza})
\end{itemize}

\section{Návod na lokálne spustenie}
\label{navod}

V~tejto podkapitole si povieme ako spustiť aplikáciu lokálne. Návod sa spolieha, že sú splnené všetky predpoklady z~podkapitoly~\ref{predpoklady}, a~takisto na to, že čítateľ už má k dispozícií repozitár aplikácie (jednou z~možností ako získať repozitár aplikácie je stiahnuť si ho z~online repozitára, pre odkaz naň viď prílohu~\ref{implementacia}).

\begin{enumerate}
  \item V priečinku (repozitári) aplikácie si dvojitým kliknutím na~\verb|ServIS.sln| otvoríme aplikáciu v programe Visual Studio 2022.
  \item V \verb|appsettings.json| sa nachádza konfigurácia aplikácie, kde je možné nastaviť napr.~názov firmy, aký má byť minimálny rozdiel medzi ponukami v~aukcii atď. Nastavíme ju podľa svojich údajov (\verb|Logging| a~\verb|AllowedHosts| môžeme ignorovať).
  \newpage
  \item Pravým tlačidlom myšy klikneme na projekt ServISData, a~potom klikneme na~\uv{Manage User Secrets}. Zobrazí sa nám súbor, ktorého obsah by mal vyzerať takto:
  
\begin{verbatim}
{
    "ConnectionStrings": {
        "Default": "CONNECTION_STRING"
    }
}
\end{verbatim}  
  
Namiesto \verb|CONNECTION_STRING| vložíme svoj connection string databázy (stačí ak využijeme štandardný tvar connection stringu~\cite{standard connection string}).

  \item Pravým tlačidlom myšy klikneme na projekt ServISWebApp, a~potom klikneme na \uv{Manage User Secrets}. Zobrazí sa nám súbor, ktorého obsah by mal vyzerať takto:
  
\begin{verbatim}
{
    "SyncfusionLicenseKey": "LICENSE_KEY",
    "EmailAppPassword": "APP_PASSWORD"
}
\end{verbatim}  
  
Namiesto \verb|LICENSE_KEY| vložíme svoj licenčný kľúč (postup získania kľúča je opísaný v~dokumentácii spoločnosti Syncfusion~\cite{license key}). Namiesto \verb|APP_PASSWORD| vložíme svoje emailové heslo aplikácie (postup získania emailového hesla aplikácie je opísaný na stránkach podpory spoločnosti Google~\cite{app password}).
  
  \item Kliknutím na zelený play button v hornej časti Visual Studia sa program preloží a~spustí. Po nejakej chvíli by sa nám malo zapnúť okno internetového prehliadača s našou aplikáciou.
\end{enumerate}

\section{Nasadenie do Azure}

Čo sa týka reálneho nasadenia na internet, tak jednou z~populárnych možností je nasadenie do cloudovej platformy~Azure spoločnosti~Microsoft~\cite{azure}. Aplikácia napísaná pomocou frameworku~Blazor Server sa spúšťa z~ASP.NET~Core aplikácie, preto je jej nasadenie podobné ako v prípade ASP.NET Core aplikácie~\cite{blazorserveraspnet}. Kvôli tomu je možné pre~nasadenie našeho systému využiť návod na~nasadenie ASP.NET aplikácie s~databázou opísaný v~dokumentácii~\cite{deploy} s~menšími obmenami~-- na~stránke~\uv{Create Web App + Database} si namiesto možnosti~\uv{Azure SQL Database} vyberieme \uv{MySQL~-- Flexible Server}, a~takisto v~časti \uv{Configuration} vyplníme potrebné kľúče s hodnotami, ktoré boli spomenuté v~krokoch 2., 3. a~4. v predošlej podkapitole (vid \ref{navod}). Navyše je možné preskočiť prácu s Azure Cache for Redis, pretože ju v programe nijako nevyužívame.
